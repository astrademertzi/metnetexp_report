\section{Conclusions \& Outlook}
\normalsize

This research project is the first to explore the expansion of metabolic networks constrained by phylogenetic relationships, and to investigate the effect of taxon sampling on ancient genome inference. 

Phylogenetic analyses appear robust to random taxon sampling for the same dataset size. While including more species can better capture the evolutionary relationships between genes and gene families, it does not necessarily enhance species tree accuracy, as previously stated \cite{martinez-gutierrez2021}. These results suggest that depending on the research question, the inclusion of more species may be beneficial. For instance, studies aiming to define Archaeal or Bacterial Tree topologies will benefit from more limited taxon sampling, while studies like this one need broader sampling to capture the full extent of prokaryotic diversity. 

Results can vary significantly depending on both taxon sampling and the evolutionary model used for ancient gene content inference. Unlike previous research, which has investigated the physiology and metabolic potential of prokaryotic ancestors such as LUCA, LACA, and LBCA, we have employed a different parsimonious evolutionary model, the DCL model. This model does not account for horizontal gene transfer, which is integral to prokaryotic life. Consequently, the inferred genome sizes correlate positively with the number of species included in the analysis, with ancient gene content being multiple times larger than the average extant genome for any given dataset of the analysis. The DCL model, however, remains relevant. By shifting focus from the hypothesis that a single microorganism gave rise to all extant diversity to the possibility that a pool of microorganisms from the ancient biosphere evolved into modern life, the DCL model can provide valuable insights into the core metabolic functions of extant life.

Regarding the reconstruction of both extant and ancient metabolic networks using eggNOG-mapper, a gold standard and state-of-the-art tool for gene functional annotation, we found that a portion of annotated genes expected to be assigned a KEGG function were not assigned one. This finding underscores two points of interest in bioinformatics research: the discrepancy of gene family clustering algorithms and function assignment (KOs, COGs, eggNOGs) between the various tools and databases used in genomic and metagenomic workflows, and the need for transitioning to a more comprehensive database than KEGG for metabolic modeling. 

As far as the evolution of metabolism is concerned, conclusions are less concrete. The current analysis points to the existence of all six enzyme categories in the ancient biosphere. Their relative abundance, however, was probably different to that of extant life.

Performing metabolic network expansion with various seed sets across the extant and extinct parts of the tree of life challenges the previously held view that this process reflects evolution-like traits. We show that the expansion solely depends on the nature, topology, and number of seeds used for expansion, and echos the flow of chemical information within the network. Different seeds will yield differentially expanded networks, which highlights the importance of nutrient availability for proper metabolic function. 

The future of this research project is full of exciting possibilities. An important development is the creation of a \textit{'poor species detection'} algorithm. This tool will automate the identification of genetically underrepresented species in a dataset and enable the gradual integration of new closely-related taxa. The aim is to improve initially poor taxon sampling over time, while keeping the dataset size minimal, making the process both time- and cost-effective. This approach may also estimate under-sampled species in the tree of life, facilitating future sampling efforts.

Another potential improvement lies in the evolutionary analysis of metabolism we have performed. Instead of analyzing EC evolution across the tree of life, it will be more informative to focus on enzyme evolution within specific clades. This approach will allow for a more detailed understanding of the evolution of metabolic pathways for specific groups of microorganisms with specialized metabolisms, such as methanogens, halophiles, or thermophiles.

The current study has only scratched the surface of what is possible with regard to metabolic network expansion. A very useful extension of the analysis would be to use MNE to quantify a system's metabolic potential based on metabolite availability in the environment, essentially by developing a metric to measure expansion. Since our current, randomly generated seed sets are sometimes poorly integrated into the reconstructed metabolic networks, it would be beneficial to create manually curated seed sets of comparable size and complexity. This would allow for a more detailed investigation of the flow of chemical information within specific networks of interest. 

The most valuable addition to the MNE analysis would be integrating the minimal seed set approach. This method would help identify the nutritional requirements of putative ancient microorganisms and provide insights into the environmental conditions they may have inhabited.

Despite its exploratory nature, this project provides valuable insights into the challenges of modern phylogenetics and introduces a new perspective on the nature of MNE. Overall, these observations underscore the importance of delving deeper into the data, questioning it, and maintaining an open mind when investigating the deepest branches of the tree of life.