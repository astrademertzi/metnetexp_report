\section{Introduction}
\normalsize

\subsection*{A surge in available data}
\addcontentsline{toc}{subsection}{A surge in available data}

The advancement of sequencing technologies and the development of state-of-the-art bioinformatics tools have resulted in a rapid increase not only in the number of sequenced genomes, but also in the sampled prokaryotic diversity, expanding our view of the tree of life, and especially that of the archaeal branch. This surge in data availability can readily improve the resolution of modern phylogenetic relationships, resolve further their deep evolutionary history, and provide more accurate inference of ancestral gene contents \cite{tahon2021}. The first release of GTDB back in 2019, R89, contained 145,904 genomes organized into 24,706 species clusters, while its latest, R220 release this April, has grown to contain 596,859 genomes organized into 113,104 species clusters \cite{parks2018, parks2020, parks2022, rinke2021}; an increase of more than 300\% in just five years.

To date, several studies have investigated the nature and physiology of LACA, LBCA, and LUCA (Last Universal Common Ancestor) through gene content estimation \cite{williams2017, xavier2021, moody2024, coleman2021}, with datasets of various sizes and microbial physiologies, originating from a number of different databases. Optimally these datasets are constructed for tree rooting and take into account taxon sampling evenness, yet it is still unclear how the incorporation of more species or other species belonging to the same higher taxonomic level will affect the analysis.
 
Moreover, GTDB, in its effort to standardize prokaryotic taxonomy, employs rank-normalization for the classification of microorganisms at all taxonomic levels \cite{parks2018}, which leads to periodic reclassification and renaming of taxa in the database. The change in nomenclature and the constant taxonomic revisions create confusion even when referencing taxa from the same database, and hinder the refinement of deep phylogenetic relationships \cite{tahon2021}. Especially in an era of exponential growth in genomic data, there is a pressing need to determine the robustness of standard methodologies and result stability.


\subsection*{A misconception about Metabolic Network Expansion}
\addcontentsline{toc}{subsection}{A misconception about Metabolic Network Expansion}

Metabolic Network Expansion (MNE) is a heuristic developed by Ebenhoh et al. (2004) \cite{ebenhoh2004} focused on analyzing the topology of large-scale metabolic systems (see \textit{Brief Explanations} sections for how expansion is performed). Since its inception, MNE has found many applications, including the investigation of the biosynthetic potential of a metabolite \cite{handorf2005, ebenhoh2006a}, the study of symbiotic relationships and interactions within microbial communities \cite{christian2007, frioux2018}, and the characterization of the minimal nutritional requirements of diverse microorganisms \cite{handorf2008,borenstein2008}.

While MNE does not account for evolutionary constraints, and has not been used within a phylogenetic framework, the expansion has been suggested to display characteristics relevant to the evolution of metabolism \cite{handorf2005, ebenhoh2006a, kreimer2008} by revealing the \textit{'temporal order of the emergence of metabolic pathways'} \cite{handorf2005}. Given that these claims are based on the sequence of reaction additions during the expansion, which are inherently dependent on the network seed compounds, it is crucial to reexamine the nature of the MNE algorithm and explore the evolution of metabolism from a phylogenetic perspective.


\subsection*{Brief Explanations}
\addcontentsline{toc}{subsection}{Brief Explanations}

\subsubsection*{The Advantage of OrthoFinder, and a dive into Orthology.}
\addcontentsline{toc}{subsubsection}{The Advantage of OrthoFinder, and a dive into Orthology}
To understand the evolution and diversity of Earth, we need to determine the phylogenetic relationships between gene sequences. The first step of comparative genomics methodologies includes a pairwise sequence alignment of all genes of every species included in the analysis. The scores provided by these alignments are then used to define gene orthology relationships; essentially, they cluster genes into groups of similar functionality and form gene (or protein, if only protein-encoding genes are used) families. It is extremely important to take sequence length into account when clustering sequences, because longer ones tend to have higher alignment scores, and the worst scores of long sequences may be higher than the best scores of short ones. This can lead to many missing genes in OGs with short genes, and numerous incorrectly clustered genes in OGs with long genes. One major advantage of OrthoFinder over other orthology inference tools is its ability to correct for gene length bias in pairwise sequence alignment scores, allowing it to uncover missing phylogenetic relationships \cite{emms2015, emms2019}.

To understand orthology, and the various sister terms, it is important to understand the concept of homology. Put simply, biological homology is the similarity between two structures due to shared ancestry. Genes that descend from a shared ancestor are therefore homologous. Homology can be further divided into orthology, paralogy. Orthology refers to genes that diverged due to a speciation event, while paralogy refers to genes that diverged due to a duplication event. Orthology is particularly useful in comparative genomics, as it allows for the inference of gene function, because orthologs are more likely to have the same function than paralogs \cite{gabaldon2013a}. 

In phylogenetic analyses, orthologs, by definition, are \underline{pairs of genes} that descended from a single gene in the Last Common Ancestor (LCA) of the \underline{pair of species} being compared. They can be thought of as \textit{equivalent genes} between species \cite{emms2015}. OrthoFinder infers both orthology and paralogy relationships between genes, and groups them into what are called OrthoGroups (OGs). OGs are an extention of this concept to multiple species, and can thus be defined as a \underline{group of genes} descended from the LCA of a \underline{group of species} \cite{emms2015, emms2019}. 

\subsubsection*{DLC vs. DTL}
\addcontentsline{toc}{subsubsection}{DLC vs. DTL}
Gene trees can be inferred based on the assumed gene orthology relationships. These gene trees, however, have different topologies compared to species trees. To capture the true (or rather as close to the truth as possible) evolutionary relationships of gene families, gene tree-species tree reconciliation needs to be performed \cite{swenson2012}. This process helps us to better understand the evolutionary events that have influenced the history of gene families, like gene gains, losses, and duplications \cite{bansal2012}. More complex phenomena, like horizontal gene transfer and incomplete lineage sorting can also cause incongruence \cite{rasmussen2012} 

Both types of trees depict evolutionary relationships: gene trees represent the relationships within single gene families, while species trees capture the relationships among all species in the dataset under analysis. Various models exist for addressing incongruence (discrepancies) bewteen genes and species trees. Two of the evolutionary models used for reconciliation that are of interest to this study are the Duplication-Loss-Coalescence (DLC) and the Duplication-Transfer-Loss (DTL) models. As their names suggest, the DTL model accounts for gene duplication, transfer, and loss events, while the DLC model addresses gene duplication, loss, and coalescence. Since both transfer and coalescence significantly impact evolution, incorporating them into evolutionary models is crucial.

Horizontal gene transfer is a well-known phenomenon in the world of prokaryotes, a powerful process that allows for the rapid acquisition of new genes and functions on demand \cite{goldenfeld2007, boto2010}, and its integration in evolutionary models has been crucial for studies of prokaryotic evolution \cite{bansal2012}. Coalescence, also known as incomplete lineage sorting, occurs when a population undergoes multiple speciation events within a short time frame. This rapid succession of speciations can lead to variations within the same gene, known as polymorphisms; these are then randomly distributed among descendant lineages, which can result in tree incongruence \cite{rasmussen2012}.

Each model has its own strengths, and although this analysis focuses on prokaryotic genomes, we employ the DLC model because it is integrated into the OrthoFinder pipeline. In the \textit{Results \& Discussion} section, we delve into why this model offers intriguing insights, and explore what these findings might imply for microbial evolution.


\subsubsection*{How Metabolic Network Expansion Works}
\addcontentsline{toc}{subsubsection}{How Metabolic Network Expansion Works}

Before starting the expansion, a seed set must be chosen. This seed set consists of compounds within the network from which the expansion begins. During each expansion step, available compounds react with each other. If all necessary reactants are present, the reaction proceeds, and its products are added to the network. In the next iteration, these products become part of the network and are available for further reactions. This process continues until no new reactions can proceed. The compounds in the expanded network after the final iteration are referred to as the \textit{scope} \cite{ebenhoh2004, handorf2005}.